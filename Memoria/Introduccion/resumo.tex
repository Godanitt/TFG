\thispagestyle{empty} %%<- poner en mas paginas si los resumenes ocupan varias (elimina la numeracion de la pagina actual)
\pagebreak


\begin{flushleft} \hyphenpenalty=10000\exhyphenpenalty=10000 {\bf $\bullet$ Resumo:\;\;}
%
Este traballo céntrase no estudo da función de onda do estado fundamental do \litioOnce e no estudo da estructura do núcleo \litioDiez mediante a reacción de trasferencia $^{11}\text{Li}(d,t)^{10}\text{Li}$. Esta reacción permite determinar as distintas proporcións dos orbitais $2s_{1/2}$ e $1p_{1/2}$ presentes no estado fundamental do \litioOnce e medir a espectroscopía do \litioDiez. Empregouse o detector ACTAR TPC, que permite rexistrar con gran precisión as trazas das partículas grazas ao seu funcionamento en cinemática inversa e ao uso dun gas activo que actúa como obxectivo e medio de detección. Ademais, dispóñense muros de detectores de silicio que permiten medir a enerxía residual das partículas saíntes, clave para reconstruír a enerxía de excitación de $^{10}\mathrm{Li}$. A simulación desenvolvida, baseada en métodos de Monte-Carlo, tivo en conta efectos como o straggling, a resolución enerxética dos detectores e a resolución angular. Os resultados obtidos mostran que a resolución angular é a principal fonte de incerteza, limitando a capacidade de separar claramente os estados $2s_{1/2}$ e $1p_{1/2}$. Mellorar esta precisión permitiría unha caracterización máis detallada dos estados resonantes de $^{10}\mathrm{Li}$ e, por extensión, do halo de $^{11}\mathrm{Li}$. Tamén vemos perspectivas de mellora da estadística a través da recostrucción de trazas co detector ACTAR TPC. Este estudo contribúe á mellora dos modelos fenomenolóxicos que describen núcleos exóticos preto da dripline, fundamentais para comprender os procesos de nucleosíntese de elementos pesados no universo.


%
\end{flushleft}\mbox{}


\begin{flushleft} \hyphenpenalty=10000\exhyphenpenalty=10000 {\bf $\bullet$ Resumen:\;\;}
%
Aqu\'{\i} va el resumen en castellano. El orden de los idiomas puede cambiarse a voluntad. Tambi\'en (y aunque la hoja de res\'umenes no contabiliza para el n\'umero l\'{\i}mite de p\'aginas, ver secci\'on \ref{quecontabiliza} de este documento) puede  reducirse, si se desea, el tama\~no de letra de los res\'umenes en los dos idiomas que no sean el usado  en el texto principal de la memoria (por ejemplo, anteponiendo \verb \footnotesize{}  al texto). % anteponiendo \footnotesize{} al texto, por ejemplo
%
\end{flushleft}\mbox{}


\begin{flushleft} \hyphenpenalty=10000\exhyphenpenalty=10000 {\bf $\bullet$ Abstract:\;\;}
%
The abstract goes here.
%
\end{flushleft}\mbox{}

\begin{center}
\rule{61mm}{0.1mm}\\
\end{center}
{\bf Agradecementos (opcional)}
Aquí pódense incluir os agradecementos, se non foron inclu\'{\i}dos xa na portada interior.
