
\section{Conclusiones}


En este trabajo se ha estudiado el \litioDiez y \litioOnce desde una perspectiva teórica y experimental, basada en el modelo de capas y aplicando conocimientos teóricos y resultados experimentales previos. 

Tras esto, caracterizamos la interacción $^{11}\text{Li}(d,t)^{10}\text{Li}$ mostrando sus ventajas frente otras a la hora de estudiar la estructura del núcleo \litioOnce, un tipo de núcleo halo. Por todas sus ventajas es una de las interacciones que se pondrán a punto en el detector ACTAR TPC, el cual también explicamos. En este contexto tratamos de simular con modelos teóricos de sección eficaz y conocimientos experimentales de los estados de dichos núcleos los resultados obtenibles con este dispositivo a través de un complejo algoritmo escrito en \verb|C++| y específicamente diseñado para ACTAR TPC. 

La simulación puso entonces a prueba los posibles resultados de la reconstrucción de la energía de excitación incluyendo ciertos modelos para las fuentes de incertidumbre conocida, llegando a la conclusión de que el factor que más afecta es la resolución angular, y, en menor medida, el \textit{straggling} energético. Pudimos obtener una $\sigma$ con un valor similar a la separación de las energías de excitación cuando tenemos en cuenta todas las fuentes de incertidumbre, y fuimos capaces de reproducir las anchuras $\Gamma$ sampleadas con bastante precisión. Las $\sigma$ aquí obtenidas permitirán determinar las anchuras $\Gamma$ a la hora de analizar el experimento real, ya que podrán ser usadas como parámetro fijo en el ajuste \textit{Voigt} a los datos experimentales dado que ahora será $\Gamma$ el parámetro libre. 

Por último, vimos como recuperar los ángulos y energías más bajas no abarcados por los silicios gracias al \textit{trigger L1}, recuperando una gran cantidad de eventos que, pese a que por el momento no se ha incluído su contribución a la resolución en la simulación, podrían permitirnos mejorar la estadística y quizás poder obtener picos de excitaciones más poblados. Los datos obtenidos aquí nos sugieren que podríamos recuperar entorno a un 30\% para $E_{ex}=0.0$ MeV  y un 60\% para $E_{ex}=0.2$ MeV de la estadística de baja energía.

