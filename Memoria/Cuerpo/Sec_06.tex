\section{Conclusiones}


En este trabajo se ha estudiado el litio 10 y litio 11 desde una perspectiva teórica, basada en el modelo de capas aplicando conocimientos teóricos y resultados experimentales. 

Tras esto caracterizamos la interacción $^{11}\text{Li}(d,t)^{10}\text{Li}$ mostrando sus ventajas frente otras a la hora de estudiar la estructura de los núcleos halo. Por todas sus ventajas es una de las interacciones que se podrán a punto en el nuevo detector ACTAR TPC que también explicamos. En este contexto tratamos de simular con modelos teóricos de sección eficaz y conocimientos experimentales de los estados de dichos núcleos los resultados obtenibles con este nuevo detector a través de diferentes progaramas en C++ Root, y ActPhysics, un código especializado en simulaciones y tratamiento de datos de este detector. 

La simulación puso entonces a prueba los posibles resultados de la recostrucción de la energía de excitación incluyendo ciertos modelos para algunas fuentes de incertidumbre conocida, llegando a la conclusión de que el factor que más afecta es el straggling angular, y luego el straggling energético. Existen errores sistemáticos procedentes de la simulación que se comprobo que eran insignificantes con los anteriores. También se llego a la conclusión de que la mezcla de una anchura $\Gamma$ relativamente grande considerando el actual sttragling angular hace que no distingamos los picos de las excitaciones lo que es un punto negativo del experiemento. 

Por último, vimos como recupear los ángulos y energías no abarcados por los silicios gracias al trigger L1, recuperando una gran cantidad de eventos que, pese a no estar implementados, podrían permitirnos mejorar la estadística y quizás poder obtener picos de excitaciones mucho más pequeñas.


\begin{table}[H] \centering 
\begin{tabular}{@{}llll@{}}
\toprule
                      & $\sigma(0.0)$ {[}MeV{]}              & $\sigma(0.20)$ {[}MeV{]}             &  \\ \midrule
$\sigma_{tot}$        & $\num{0.218965419142(0.0006862477)}$ & $\num{0.180449369562(0.0010214727)}$ &  \\
$\sigma_{straggling}$ & $\num{0.086197097183(0.0004708945)}$ & $\num{0.065691148826(0.0011537355)}$ &  \\
$\sigma_{\theta}$     & $\num{0.201360910124(0.0006740057)}$ & $\num{0.170298615966(0.0010107562)}$ &  \\
$\sigma_{0}$          & $\num{0.001363305730(0.0000960943)}$ & $\num{0.011362236853(0.0034089820)}$ &  \\ \bottomrule
\end{tabular}
\caption{$\sigma$ del ajuste gaussiano a la distribución de energía de excitación recostruida.}
\label{Tab:05-ExcRec}
\end{table}