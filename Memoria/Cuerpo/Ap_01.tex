
\section{Convenciones de Nombres}

En C++, seguir convenciones claras de nombres para las variables, funciones y clases facilita la legibilidad y el mantenimiento del código. A continuación, se presentan las prácticas más comunes en la denominación de distintos elementos.ç
\subsection{Clases}
Las clases suelen seguir la \textbf{Convención PascalCase}, donde cada palabra comienza con mayúscula y no se usan guiones bajos.

\begin{itemize}
\item Ejemplo: \texttt{Particle}, \texttt{CollisionEvent}, \texttt{EnergyCalculator}
\end{itemize}

\subsection{Variables}
\begin{itemize}
\item \textbf{int}: suele usarse \textbf{camelCase}, empezando con minúscula y sin guiones bajos.
\begin{itemize}
\item Ejemplo: \texttt{eventCounter}, \texttt{particleId}
\end{itemize}
\item \textbf{double}: se sigue el mismo esquema \textbf{camelCase}, pero indicando la magnitud si es relevante.
\begin{itemize}
\item Ejemplo: \texttt{beamEnergy}, \texttt{collisionAngle}
\end{itemize}
\item \textbf{bool}: comienza con \texttt{is}, \texttt{has}, \texttt{can}, etc., seguido de PascalCase.
\begin{itemize}
\item Ejemplo: \texttt{isActive}, \texttt{hasCollided}
\end{itemize}
\end{itemize}

\subsection{Constantes}
Las constantes globales o locales usan \textbf{UPPERCASE WITH UNDERSCORES}.

\begin{itemize}
\item Ejemplo: \texttt{PI}, \texttt{MAX PARTICLES}, \texttt{SPEED OF LIGHT}
\end{itemize}

\subsection{Variables que pueden cambiar (mutable)}
Siguen las convenciones generales para \textbf{int} o \textbf{double}, sin notación especial adicional:

\begin{itemize}
\item Ejemplo: \texttt{currentEnergy}, \texttt{numberOfSteps}
\end{itemize}

\subsection{Funciones y Métodos}
Se utiliza \textbf{camelCase}, iniciando en minúscula.

\begin{itemize}
\item Ejemplo: \texttt{calculateEnergy()}, \texttt{setParticleType()}
\end{itemize}

\subsection{Miembros privados de clase}
Se puede usar un guion bajo al final o al principio para diferenciarlos.

\begin{itemize}
\item Ejemplo: \texttt{mass}, \texttt{energy}
\end{itemize}

\subsection{Resumen de estilos}
\begin{tabular}{|c|c|}
\hline
Tipo & Convención de Nombre \\
\hline
Clases & PascalCase \\
Variables (int, double) & camelCase \\
Booleanos & camelCase con prefijos (is, has) \\
Constantes & UPPERCASE WITH UNDERSCORES \\
Funciones & camelCase \\
Miembros privados & camelCase con guion bajo al final o inicio \\
\hline
\end{tabular}