\section{Motivación experimental}

\subsection{Núcleos ricos en neutrones}

El estudio de núcleos ricos en neutrones son, en la física nuclear moderna, una de las ramas de mayor interés. No solo porque son fundamentales a la hora de entender los procesos r astrofísicos \cite{THIELEMANN2011346}, (proceso de captura rápida de neutrones), los cuales parecen responsables de la creación de la mayor parte de núcleos muy pesados $(60<A)$, que se dan en la expansión tras un colapso del núcleo de una supernova, o la descompresión de la materia neutrónica emitida por la fusion de una estrella binaria compacta de neutrones; sino porque revelan nuevas estructuras nucleares (halos nucleares, modificación del orden de las capas nucleares y números mágicos...). 

Estas nuevas estrucutras nucleares ponen a prueba modelos nucleares a prueba, y arrojan información necesaria para adaptar modelos fenomenológicos que permitirán obtener resultados teóricos extrapolados a otros neutrones ricos en neutrones imposibles de medir experimentalmente por la incapacidad actual de producirlos (debido a su baja estabilidad y su falta de estados ligados).



\subsection{Núcleo halo}

La drip-line de neutrones sigue siendo un misterio para nosotros, conociendo únicamente 8 o 9 elementos sobre ella, particularmente en los átomos ligeros. En aquellos que se ha alcanzado esta drip-line los núcleos presentan formas exóticas, comportamientos anómalos que no son vistos en núcleos en la estabilidad. Los halos de neutrones son uno de los fenómenos más llamativos, ya que presentan propiedades inesperadas, como un tamaño (radio) mas grande de lo que se habría (\textcolor{red}{{{Poner ejemplo}}}). \cite{tanihata2023halo}.

El halo de neutrones es, en esencia, una manifestación del efecto túnel cuántico, que surge cuando un estado nuclear ligado se encuentra muy próximo al continuo energético  \cite{tanihata2023halo}, es decir, a un estado no ligado. Para que se de una estructura de halo se necesita una combinación de energía de ligadura de neutrones muy pequeña  ($<1$ MeV) y una fuerza de corto alcanzce (como es la fuerza nuclear). El requerimiento de que la energía de ligadura sea pequeña hace que la mayor parte de los halos solo puedan tener uno o dos neutrones en el halo. Dicha combinación de factores permite al neutrón a través del efecto túnel moverse alrededor del core nucleo, lo que conlleva a su vez que el núcleo tenga un tamaño más grande de lo normal: la función de ondas de los neutrones hace probable que este a distancias mucho más lejanas. 

Un núcleo Borromeano es un sistema cuántico de tres cuerpos que se representa como un núcleo con halo de dos neutrones (2n), compuesto por un núcleo central + n + n, en el que ninguno de los subsistemas de dos cuerpos (núcleo + neutrón o neutrón + neutrón) está ligado, pero el sistema completo de tres cuerpos sí forma un estado ligado \cite{tanihata2023lowEnergyHalo}. Ejemplos típicos de átomos Borremeanos: $^6$He, $^{11}$Li, $^{14}$Be y $^{17}$B. 

\subsection{Reacción de traferencia $^{11}$Li(d,t)$^{10}$Li}

La reacción en la que nos vamosa centrar nosotros es: 

\begin{equation}
   {}^{11}\text{Li} + d \to t + {}^{10}\text{Li}
\end{equation}

Esta reacción es una de las más interesantes a la hora de obtener información acerca de núcleos halo con precisión, en particular información acerca el $^{10}$Li. 
La reacción de transferencia \({}^{11}\text{Li}(d,t){}^{10}\text{Li}\) es particularmente interesante dentro del estudio de núcleos exóticos debido a su capacidad para proporcionar información directa sobre la estructura del halo de \({}^{11}\text{Li}\). A diferencia de muchas reacciones que sólo permiten estudiar el espectro excitado de \({}^{10}\text{Li}\), esta reacción accede directamente a las configuraciones de un solo neutrón en \({}^{11}\text{Li}\), facilitando la reconstrucción de sus componentes estructurales.


¿Por qué es relevante estudiar \({}^{11}\text{Li}(d,t){}^{10}\text{Li}\)?
\begin{itemize}
    \item Permite investigar el papel de los estados resonantes de \(^{10}\text{Li}\) en la estructura del halo de \(^{11}\text{Li}\). Dado que $^{10}\text{Li}$ es inestable y no tiene un estado ligado, su estudio experimental es muy complicado, y esta reacción permite observar sus resonancias de forma más directa. \cite{SANETULLAEV2016481}.
    
    \item La transferencia de un neutrón desde \(^{11}\text{Li}\) al deuterón que forma el tritón permite poblar estados de \(^{10}\text{Li}\) con características específicas de momento angular y energía, mostrando la naturaleza de las configuraciones \(s_{1/2}\) y \(p_{1/2}\) en el estado fundamental de \({}^{11}\text{Li}\).  \cite{CASAL2017307}. 
    
    \item Es sensible a los factores espectroscópicos, es decir, permite medir la probabilidad de encontrar una cierta configuración de un neutrón y \({}^{10}\text{Li}\) en el estado fundamental de \({}^{11}\text{Li}\), lo cual no es accesible en muchas otras reacciones. \cite{SANETULLAEV2016481}. 
\end{itemize}

Ventajas frente a otras reacciones de transferencia
\begin{itemize}
    \item A diferencia de otras como \({}^{9}\text{Li}(d,p){}^{10}\text{Li}\), la reacción \({}^{11}\text{Li}(d,t){}^{10}\text{Li}\) accede directamente a la estructura del halo en \({}^{11}\text{Li}\), no sólo a la existencia de resonancias en \({}^{10}\text{Li}\). \cite{CASAL2017307}. 
    
    \item El modelo DWBA aplicado en este tipo de estudios, junto con funciones de solapamiento derivadas de un modelo tridimensional, permite una comparación más directa y realista con datos experimentales. \cite{CASAL2017307}.
\end{itemize}


En resumen, la reacción \({}^{11}\text{Li}(d,t){}^{10}\text{Li}\) no sólo aporta datos sobre los estados de \({}^{10}\text{Li}\), sino que se convierte en una herramienta crucial para entender cómo se configura el halo en \({}^{11}\text{Li}\), permitiendo extraer porcentajes de contribuciones tipo \(s_{1/2}\) y \(p_{1/2}\), y estudiar la ruptura del cierre de capa en \(N=8\). Por ello, tiene una relevancia singular dentro de la física nuclear de sistemas exóticos.



\subsection{Modelo de capas}