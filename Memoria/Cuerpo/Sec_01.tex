

\section{Introducción}


La física nuclear es una rama fundamental de la física que estudia las propiedades, estructura y transformaciones del núcleo atómico. Desde el descubrimiento de la radiactividad a finales del siglo XIX hasta el desarrollo de modelos nucleares avanzados en el siglo XXI, esta disciplina ha tenido un profundo impacto tanto en la comprensión de la materia como en múltiples aplicaciones tecnológicas y científicas. Entre estas destacan la producción de energía nuclear, las técnicas de diagnóstico y tratamiento médico, la datación radiométrica y el estudio de procesos astrofísicos como la nucleosíntesis estelar.

Los núcleos con un alto número de neutrones, según los modelos actuales, juegan un papel fundamental en los procesos astrofísicos, en partiuclar en la producción de átomos pesados que hallamos en el unvierso. En este contexto el estudio de sus características principales se hacen fundamentales para comprender cómo funcionan estas cadenas de producción de átomos pesados, tanto para comprender bajo que coindiciones aparecen y conocer cuales reacciones predominan y en qué clase de estrellas.

Sin embargo estudiar átomos ricos en neutrones presenta una gran dificultad, al ser núcleos muy poco estables e incluso no ligados. Precisamente por la difiucultad de producción en muchos casos se necesitan instalaciones y experimentos muy partiuclares, diseñados específicamente para medir reacciones nucleares. Entre estos experimentos en,ontramos en ACTAR TPC con implicación directa de la USC a través del IGFAE, y que como veremos, nos permitirá caracterizar con precisión reacciones nucleares de trasferencia, qué tal y como veremos, nos peritirá extraer información acerca de núcleos con un número alto de neutrones. 

